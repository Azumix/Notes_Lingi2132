\chapter{Pumping lemma}
\label{chap:pumping_lemma}

\section{Recap}
\theoremstyle{definition}
\begin{definition}[Grammar]
    Grammar = ($N, \Sigma, P, s$)
    \begin{itemize}
        \item N : Non-terminals
        \item $\Sigma$: Alphabet (terminals)
        \item P: Production rules
        \item Starting Symbol (S $\in$ N)
        \item $P: N \rightarrow \Sigma * N?$
    \end{itemize}
    Matchable by a DFA!
\end{definition}

Regular languages are capable of expression repetition and optionality thanks to
the optional final non-terminal. However, it can't do nesting.

We can ask ourself : how to know if a language is regular ? 
And how to prove (just build the grammar, is that match it works)/disprove (pumping lemma) it?
\section{Central Recursion}
    Let's take an example of non-regular language: $A ::= '('A')' | \epsilon$.
    This is called central recursion. Note that language with central recursion
    are never regular. That's why C(or Java) does not have nested comments

    A way to check if a language is regular is to check if it is matchable by a
    DFA (that has only one piece of state, the current one). If we want to match
    the central recursion, we need an extra counter for the depth. It exists
    only one way to implement that : using non-terminal in the middle. The
    problem is : this is not allowed in regular languages. We can proof it using
    the pumping lemma.
\newpage
\section{The Pumping Lemma}
    \theoremstyle{definition}
    \begin{definition}[Pumping Lemma]
        If L is a regular language:
        \begin{itemize}
            \item Then it exists an integer $p\geq 1$ specific to L
            \item Such that every sentence s in L of $length \geq p$ can be
            written $s = x y z$ satisfying the following conditions :
            \begin{itemize}
                \item $|y| \geq 1$
                \item $| x y| \leq p$
                \item $\forall n \geq 0, x y^n z \in L$
            \end{itemize}
        \end{itemize}

        Every long enough ($p\geq 1$ and specific to L) string in the language
        can be used to generate longer string through a repeating part (y).

        Note that there is also a corollary : every long enough string can be
        decomposed into a prefix, a suffix and a repeating part.
    \end{definition}

    Thanks to the pumping lemma, we can proof that $L = (^k )^k$ is not regular
    (using contraction). Indeed, as we can show that $x$ and $y$ have to be made
    only of '(' and that $y$ cannot be empty by definition, if we try to repeat
    y, the parenthesis are unbalanced! Which is a contradiction.

    The pumping lemma can show us that any finite language is regular! Even a
    really simple one like $L = a | b | c$. We just need to take a big enough p!
    (Like $p=2$ in this case).

    \subsection[Reasons]{Why do we need $|x y | \leq p$ ?}
        If we omit it, we could prove that $L = (^k a+ )^k$ to be regular which
        is obviously not the case!. We need central recursion to be bounded
        because of DFA state limit. The constraint ensures we're able to build a
        maximum depth counter-example.
    \subsection{Be care}
        Note the way the pumping lemma is written, it is an implication. That
        mean $A \implies B \neq B \implies A$. Being able to pump sentences does
        not make a language regular. However is it a necessary condition in
        order to be regular.
    \subsection{In CFGs}
        The pumping lemma also work in CFGs! The difference is instead of
        splitting a sentence in 3, we split it in 5 $s = v w x y z \implies v w
        w x y y z \in L$ the parenthesis language works if we pick w=( and y=)
        note that a language like $a^k b^k c^k$ does not work. In both case, the
        pumping lemma can prove it!
    \subsection{In everything else}
        We just have to decompose in a pattern : 
        \begin{itemize}
            \item X be a set of sets (e.g regular languages)
            \item An infinite set (language) of objects (sentences) in X can be
            obtained if long object repeat some sub-elements 
            \item We can use the pumping lemma to disprove the belonging of a
            language to X
        \end{itemize}
